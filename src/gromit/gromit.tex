%*********************************************************************************************************************%
%                                                                                                                     %
%                                                    Wallace                                                          %
%                                                                                                                     %
%                               Fran�ois Pottier, projet Cristal, INRIA Rocquencourt                                  %
%                                                                                                                     %
%    Copyright 2000 Institut National de Recherche en Informatique et Automatique. Distributed only by permission.    %
%                                                                                                                     %
%*********************************************************************************************************************%
% $Header: /home/pauillac/formel1/fpottier/cvs/gromit/gromit.tex,v 1.5 2000/02/11 16:15:36 fpottier Exp $

\documentclass[a4paper]{article}
\usepackage{amsmath}
\usepackage{amssymb}
\usepackage{amstext}
\usepackage[T1]{fontenc}
\usepackage{geometry}
\usepackage[latin1]{inputenc}
\usepackage{latexsym}
\usepackage[bypages,novisiblespaces]{ocamlweb}

\special{papersize=210mm,297mm}
\geometry{a4paper,body={5.5in,9in}}

\title{Gromit}
\author{Fran�ois Pottier}

% ---------------------------------------------------------------------------------------------------------------------
% Redefine ocamlweb's sectioning commands.
%
% This is a little tricky. Nothing is done at the beginning of a .ml file, except storing its name. This allows us
% to later put its interface part and its code part at the same level, using a section for each.

\renewcommand{\ocwmodule}[1]{\section{Module #1: implementation}}
\renewcommand{\ocwinterface}[1]{\section{Module #1: interface}}

\newcommand{\mysection}[1]{\subsection{#1}}
\newcommand{\mysubsection}[1]{\subsubsection{#1}}

% ---------------------------------------------------------------------------------------------------------------------

\begin{document}
\maketitle
\newpage
\tableofcontents
\newpage

\section{Overview}

Gromit is a tool which should help you customize Wallace, the constraint-handling library, for your own needs.
Wallace is implemented as one big functor (see Wallace's \verb+Core+ module), whose most important parameter is a
``ground signature'' \verb+G+. To satisfy this signature (described in Wallace's \verb+Ground+ module), you must
provide a structure which completely specifies the set of types you intend to work with. This involves defining the
type of terms, and writing functions which iterate over terms, perform constraint propagation between terms, compute
greatest lower bounds and least upper bounds of terms, etc.

Implementing such a structure from scratch may represent quite a lot of work, and may be very error-prone, if you are
using a rich set of types. Thus, it seems better to only write a high-level description of your type algebra,
something like
%
\begin{quote}
There is a binary type constructor $\rightarrow$.\\
$\rightarrow$ is contravariant in its first argument.\\
$\rightarrow$ is covariant in its second argument.
\end{quote}
%
and to leave the work of compiling, or interpreting, this description to an automated tool. Gromit is such a tool; it
accepts a high level definition file, and compiles it. It creates an O'Caml file which implements Wallace's
\verb+Ground.Signature+. As a bonus, it also produces an \verb+ocamlyacc+ grammar which parses types.

\section{Limitations}

The syntax of Gromit's input description files is currently undocumented. Have a look at Gromit's parser in
\verb+parser.mly+. You should also have a look at the sample file included in my sample implementation, \verb+toy+;
the file is \verb+toy/mlAlgebra.grm+.

Gromit and Wallace's support for parsing and pretty-printing may be too limited for your needs. It is mainly intended
to help with prototype implementations; you may want to remove it in a production implementation.

Gromit currently requires the set of types to be fixed. This disallows user-defined types. It is possible to add a
mechanism for defining new types at runtime. You may prefer to avoid Gromit altogether and write your own code to
compile/interpret your own type definition format.

\input{gromit-auto-generated}

\end{document}


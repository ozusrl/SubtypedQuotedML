%*********************************************************************************************************************%
%                                                                                                                     %
%                                                    Wallace                                                          %
%                                                                                                                     %
%                               Fran�ois Pottier, projet Cristal, INRIA Rocquencourt                                  %
%                                                                                                                     %
%    Copyright 2000 Institut National de Recherche en Informatique et Automatique. Distributed only by permission.    %
%                                                                                                                     %
%*********************************************************************************************************************%
% $Header: /home/pauillac/formel1/fpottier/cvs/wallace/wallace.tex,v 1.7 2000/02/11 16:15:55 fpottier Exp $

\documentclass[a4paper]{article}
\usepackage{amsmath}
\usepackage{amssymb}
\usepackage{amstext}
\usepackage[T1]{fontenc}
\usepackage{geometry}
\usepackage{graphics}
\usepackage[latin1]{inputenc}
\usepackage{latexsym}
\usepackage[bypages,novisiblespaces]{ocamlweb}
\usepackage{url}

\special{papersize=210mm,297mm}
\geometry{a4paper,body={5.6in,9in}}

\title{Wallace}
\author{Fran�ois Pottier}

% ---------------------------------------------------------------------------------------------------------------------
% Redefine ocamlweb's sectioning commands.
%
% Currently (as of version 0.6), ocamlweb gives satisfactory results, except it uses \section* instead of \section,
% which prevents the sections from showing up in the table of contents.

\renewcommand{\ocwmodule}[1]{\section{Module #1}}
\renewcommand{\ocwinterface}[1]{\section{Interface of module #1}}

% Define internal sectioning commands.

\newcommand{\mysection}[1]{\subsection{#1}}
\newcommand{\mysubsection}[1]{\subsubsection{#1}}

% ---------------------------------------------------------------------------------------------------------------------
% Some useful macros.

\newcommand{\norm}[1]{\mathord\mid\, #1\,\mathord\mid}
\newcommand{\ccc}[3]{#1\leq#2\,?\,#3}
\newcommand{\cc}[4]{\ccc{#1}{#2}{#3\leq#4}}
\newcommand{\urow}{\partial}

% ---------------------------------------------------------------------------------------------------------------------

\begin{document}
\maketitle
\newpage
\tableofcontents
\newpage

\section{Overview}

Wallace is a subtyping-constraint-handling library. It allows defining, solving and simplifying subtyping constraints,
including conditional constraints. The library is parameterized by the definition of a type lattice, which may be
arbitrary, provided it is ``structural'' (i.e. two types are in the subtype relation if and only if their head
constructors are in an atomic ``sub-constructor'' relation and their common leaves are in the subtype relation).
Multiple kinds of types are supported. Rows are fully supported (see R�my's ``Projective ML'', LFP'92). Parametric
polymorphism is supported. The library performs closure operations, which allow determining whether a constraint set
has a solution. It also performs simplifications, (mostly) as described in the author's PhD thesis: canonization
(which is now performed incrementally, during closure), garbage collection and minimization (which are performed
upon request). The library includes support for converting (textual) type expressions into internal types, which
includes a check for well-kindedness and well-sortedness, and some support for pretty-printing types.

Note that the library does not implement a full type-checker; it knows nothing about the programming language which
is being analyzed, or about the way constraints should be generated. These details are up to the library's client.

The following picture gives an overview of Wallace's organization:
%
$$\scalebox{.65}{\includegraphics*{map.ps}}$$
%
As you can see, the library includes two entirely unrelated components. One is a generic library, which provides
various implementations of sets and maps, with a common interface. This allows easily replacing Wallace's internal
data structures with more efficient implementations. The other component, whose main module is \verb+Core+, contains
the constraint-handling code. Reading \verb+core.mli+ should give you an overview of Wallace's functioning. Modules
\verb+Print+, \verb+TextPrinter+ and \verb+TeXPrinter+ add some support for pretty-printing; they are of secondary
importance.

\section{Limitations}

The library is currently undocumented. The code itself contains lots of documentation, and is accompanied by a sample
program, \verb+toy+, which implements a small typechecker using Wallace. As far as theory is concerned, please have a
look at the author's thesis and papers (\url{http://pauillac.inria.fr/~fpottier/}). The implementation is rather
faithful to its theoretical description, but performs canonization incrementally (this algorithm is currently
unpublished), and has full support for rows and conditional constraints.

The library assumes that all type variables which appear within a type scheme are universally quantified. There is
no support for ``partially quantified'' type schemes, where some variables are quantified and some variables are
free. When writing a type-checker, this forces you to adopt a ``$\lambda$-lifted'' style, as I have done in my
papers. This limitation might be removed in the not-too-far future.

\input{wallace-auto-generated}

\end{document}

